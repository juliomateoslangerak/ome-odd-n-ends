\section{Future Work}

The Background Import project is, at this stage, an MRI custom extension to the OMERO
platform and, as such, not officially endorsed by OME. However, our use cases may turn 
out to be fairly common at other facilities worldwide, in which case Background Import
functionality could be integrated into mainstream OMERO. We will engage with the OMERO
core developers to discuss integration. 
This may require some changes to the Background Import code, for example, factoring it
out from the OMERO Java client into its own OMERO project. 
Also, new features could easily be added to make it more flexible and, hence, usable 
in a wider array of scenarios. Specifically, the Import Proxy could operate in ``pull''
mode too: images could be fetched from a remote workstation; thus several acquisition
workstations could share the same Import Proxy service. We estimate this to require 
roughly between one and two months of development.

More generally, one could envision a system in which end users need not worry or be
aware of where images are physically stored or how they can be accessed.
At acquisition time, an user would select where the image belongs in a given \emph{
logical} hierarchical structure of their choice---a virtual file system; under the 
hood the system would stream the image data to physical cloud storage. (Note that
physical storage and logical hierarchical structure would be \emph{independent}.)
The cloud software would then let the user access the image from any cloud-enabled
device, using the same logical hierarchical structure. OMERO would be integrated in
the cloud to enable image viewing, management, and analysis which could now be done
from any device connected to the cloud via a web browser.
We will explore this avenue further if it is of any interest to MRI.
